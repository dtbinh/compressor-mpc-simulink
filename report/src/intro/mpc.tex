\section{Distributed MPC}
\label{sec:intro:mpc}

Distributed MPC differs from standard (centralized) MPC in that the optimization problem is split into smaller sub-problems that can be solved in parallel by several distributed controllers.
This approach is particularly intuitive when dealing with large systems that combine several distinct but coupled subsystems, as is the case for a compressor network.
The sub-problems are then often associated to a particular subsystem.

Because all of the sub-problems are coupled, and the solutions to each affect the other sub-problems, distributed MPC generally takes an iterative approach whereby each sub-controller assumes a value for the other solutions, solves its optimization problem, and communicates its updated solution to the other sub-controllers.
The process can continue for a fixed number of steps, or until a convergence criterion is satisfied.
Depending on the degree of coupling, the iterations may or may not converge. For a discussion of convergence in distributed MPC controllers, see \cite{Stewart2010}.

The primary advantage of distributed MPC over centralized is the potential for reducing the computation time.
Although each of the distributed sub-controllers must solve multiple optimizations at each time step, their reduced size means that in many case they still execute faster than the centralized controller.

