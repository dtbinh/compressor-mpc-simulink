\section{Formulation of Optimization Problem}
\label{sec:mpc:optimization}

The optimization problem solved at each time step by the MPC controller is defined as follows:

\begin{equation}
  \begin{split}
    & \argmin_{\bm{U_k}}\ \Delta \bm{U}_k^\intercal\ W_u\ \Delta \bm{U}_k + \left( \Delta\bm{Y}_k - \Delta\bm{Y}\ut{ref} \right)^\intercal W_y \left( \Delta\bm{Y}_k - \Delta\bm{Y}\ut{ref} \right)\\
    \text{s.t. } & \Delta \bm{U}\ut{min} \leq \Delta \bm{U}_k \leq \Delta \bm{U}\ut{max}\\
    & \Delta \bm{Y}_k = S_{U_k} \Delta \bm{U}_k + S_{x_k} \Delta \bm{x}_k^a + S_{f_k} \bm{f}_k,
  \end{split}
\end{equation}
    
\noindent where $\Delta \bm{Y}\ut{ref}$ is the referene output, $S_{U_k}$, $S_{x_k}$ and $S_{f_k}$ are the prediction matrices giving the contribution to $\Delta \bm{Y}_k$ of $\Delta \bm{U}_k$, $\Delta x_k^a$ and $f_k$ respectively, calculated using \eqref{eq:mpc:augmented-state-eqs}, and $\Delta \bm{U}\ut{min}$ and $\Delta \bm{U}\ut{max}$ contain the input constraints (both absolute bounds and rate constraints).

The optimization is then converted to a dense formulation by eliminating the dependence on $\Delta \bm{Y}_k$ through the equality constraint, yielding the following QP problem:

\begin{equation}
  \begin{split}
    & \argmin_{\bm{U_k}}\ \frac{1}{2} \Delta\bm{U_k}^\intercal\ H\ \Delta\bm{U_k} + g^\intercal \Delta \bm{U}_k\\
    \text{s.t. } & \Delta \bm{U}\ut{min} \leq \Delta \bm{U}_k \leq \Delta \bm{U}\ut{max},
  \end{split}
  \label{eq:mpc:optimization-qp-formulation}
\end{equation}

\noindent where the QP Hessian matrix and linear term given by:

\begin{equation}
  \begin{split}
    H & = 2\left( W_u + S_{u_k}^\intercal\ W_y\ S_{u_k} \right)\\
    g & = 2\left( \Delta\bm{\hat{x}}_k^a S_{x_k}^\intercal + \bm{f_k} S_{f_k}^\intercal - \Delta \bm{Y}\ut{ref} \right)W_y S_{U_k}.
  \end{split}
  \label{eq:mpc:optimization-qp-terms}
\end{equation}

The input constraints are determined by combining limits on both the range of the inputs and on their rate of change. 
The recycle valve has a range of 0--1 with rate constraints (maximum possible change over a single sampling period) of +1/-0.1. 
The rate is more constrained in the negative direction (i.e. when closing) to prevent a transient re-entry into surge.
The torque input has a normalized range of $\pm 0.3$ compared to its steady-state value, with rate constraints of $\pm 0.1$.

