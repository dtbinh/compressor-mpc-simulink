\section{Formulation of Optimization Problem}
\label{sec:mpc:optimization}

The optimization problem solved at each time step by the MPC controller is defined as follows:

\begin{equation}
  \begin{split}
    \g{Uk} & = \argmin_{U}\ \gi{Delta} \tps{U}\ \gi{weights}\ \gi{Delta} U + \tps{\left( \gi{Delta}Y - \gi{Delta}\g{Yrefk} \right)} \gii{weights} \left( \gi{Delta}Y - \gi{Delta}\g{Yrefk} \right)\\
    \text{s.t. } & \gi{Delta} U\ut{min} \leq \gi{Delta} U \leq \gi{Delta} U\ut{max}\\
    & \gi{Delta} Y = \gi{prediction-matrices} \gi{Delta} \g{Uk} + \gii{prediction-matrices} \g{xaug} + \giii{prediction-matrices} \g{fcurr}\text{,}
  \end{split}
\end{equation}
    
\noindent where \g{Yrefk} is the reference output, \giv{prediction-matrices} are the prediction matrices giving the contribution to \gi{Delta}\g{Yk} of \gi{Delta}\g{Uk}, \g{xaug} and \g{fcurr} respectively, calculated using \eqref{eq:mpc:augmented-state-eqs}, and $\gi{Delta} U\ut{min}$ and $\gi{Delta} U\ut{max}$ contain the input constraints (both absolute bounds and rate constraints).

The optimization is then converted to a dense formulation by eliminating the dependence on \gi{Delta}\g{Yk} through the equality constraint, yielding the following QP problem:

\begin{equation}
  \begin{split}
    & \argmin_{U}\ \frac{1}{2} \tps{\gi{Delta}U}\ H\ \gi{Delta}U + \tps{g} \gi{Delta} U\\
    \text{s.t. } & \gi{Delta} U\ut{min} \leq \gi{Delta} U \leq \gi{Delta} U\ut{max},
  \end{split}
  \label{eq:mpc:optimization-qp-formulation}
\end{equation}

\noindent where the QP Hessian matrix and linear term given by:

\begin{equation}
  \begin{split}
    H & = 2\left( \gi{weights} + \tps{\gi{prediction-matrices}}\ \gii{weights}\ \gi{prediction-matrices} \right)\\
    H & = 2\left( W_u + S_{u_k}^\intercal\ W_y\ S_{u_k} \right)\\
    g & = 2\left( \g{xaug} \tps{\gii{prediction-matrices}} + \g{fcurr} \tps{\giii{prediction-matrices}} - \gi{Delta} \g{Yrefk} \right)\gii{weights} \gi{prediction-matrices}.\\
    g & = 2\left( \gi{Delta}\vc{\hat{x}}_k^a S_{x_k}^\intercal + \vc{f_k} S_{f_k}^\intercal - \gi{Delta} \vc{Y}\ut{ref} \right)W_y S_{U_k}.
  \end{split}
  \label{eq:mpc:optimization-qp-terms}
\end{equation}

The input constraints are determined by combining limits on both the range of the inputs and on their rate of change. 
The recycle valve has a range of 0--1 with rate constraints (maximum possible change over a single sampling period) of +1/-0.1. 
The rate is more constrained in the negative direction (i.e. when closing) to prevent a transient re-entry into surge.
The torque input has a normalized range of $\pm 0.3$ compared to its steady-state value, with rate constraints of $\pm 0.1$.

