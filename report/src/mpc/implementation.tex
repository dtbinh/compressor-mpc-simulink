\section{Implementation}
\label{sec:mpc:implementation}

The simulation was first implemented in \slink{} using the \texttt{ode45} solver for numerical integration of the state equations.
The MPC controllers were also implemented in \slink{} using \emlab{} scripts.
The \emlab{} syntax allows C code to be automatically generated and compiled for use in simulation or deployment on embedded systems.

Some steps were taken to decrease the number of computations required, particularly for generating the prediction matrices.% as this step was found to take over 90\% of the controller's computation time (see Section~\ref{sec:results}).
To generate these matrices, the terms $\gc{augsys-mats} {\ga{augsys-mats}}^1$, $\gc{augsys-mats} {\ga{augsys-mats}}^2$, \ldots, $\gc{augsys-mats} {\ga{augsys-mats}}^p$ must be calculated, taking the bulk of the computation time.
To speed this process up, the result of the multiplication $\gc{augsys-mats} {\ga{augsys-mats}}^i$ was saved as an intermediate variable and used to calculate $\gc{augsys-mats} {\ga{augsys-mats}}^{i+1}$.

For faster computational performance -- particularly on embedded systems -- and for a more accurate characterization of the computational efficiency of each controller, the controllers were also implemented and tested in \cpp{}. 
The \qpoases{} library presented in \cite{Ferreau2014} was used to solve quadratic programs, and linear algebra calculations were performed using the \eigen{}\cite{eigenweb} \cpp{} library.

One of the advantages of the C++ implementation compared to that in \slink{} is that it is able to use knowledge of the structure of the augmented matrices (\gv{augsys-mats}) to multiply them more efficiently.
The matrices have both dense sections, corresponding to the non-augmented states, sparse sections, corresponding to the augmented states, and sections which are zero by definition.
New types were therefore defined for each of the 3 matrices that took this structure into account. 
For \ga{augsys-mats}, for example, the matrix is as follows (with dimensions for each section for the parallel and serial systems shown in parentheses):

\begin{equation}
  \ga{augsys-mats} = 
  \kbordermatrix{%
    & \mathsmaller{(10-11)} & \mathsmaller{(2)} & \mathsmaller{(78)} & \mathsmaller{(4)} \\
    & \ga{sys-mats} & \g{Bdelay} & 0 & 0 \\[0.5em]
    & 0 &  0 & \g{Adelay} & 0\\[0.5em]
    & 0 & 0 & 0 & I_{4 \times 4}
  }
\end{equation}

The \ga{sys-mats} and \g{Bdelay} originate from the non-augmented system and are dense. 
The \g{Adelay} section, in contrast, is quite sparse: its number of nonzero entries is given by the number of total delayed states, in this case 80, minus the number of inputs that are delayed, in this case 2. 
All of the nonzero entries are furthermore equal to 1 by definition, since their purpose is to shift a delayed state up one row.
To take advantage of this structure, both the \ga{sys-mats} and \g{Bdelay} sections are implemented as dense matrices of double values, while \g{Adelay} and $I_{4\times 4}$ are combined into a single sparse matrix of booleans with one entry per column.
This modification was found to reduce the computation time required for a single controller iteration by a factor of roughly 4 compared to an implementation with fully dense matrices.

