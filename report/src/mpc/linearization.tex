\section{Linearization \& Discretization}
\label{sec:mpc:linearization}

The model used for the MPC controller formulation is the same as that used in Section~\ref{sec:modelling} to simulate the compressor.
No model mismatch was considered in order to compare the controllers' performance in an ideal scenario.
In order to generate an optimization in the form of a quadratic program, the system was linearized at each time step about the current states (\g{xhat}) and inputs (\gmo{ucurr}), and discretized using the 4th order Runge-Kutta method.
The model at sampling instant $k$ is thus given by:

\begin{equation}
  \begin{split}
    \gi{Delta} \gpio{xhat} & = \gi{sys-mats} \gi{Delta}\gpi{xhat} + \gii{sys-mats} \gi{Delta}\gpi{ucurr} + \g{fcurr}\left( \g{xhat}, \gmo{ucurr} \right)\\
    \gi{Delta} \gpi{ycurr} & = \giii{sys-mats} \gi{Delta} \gpi{xhat}, \qquad i\geq 0\\
  \end{split}
\end{equation}

\noindent where \g{fcurr} is the derivative of the system evaluated at the linearization point, \g{xhat} is the state estimate at time instant $k$, \g{p} is the prediction length of the controller, \giv{sys-mats} gives the model linearized about the current operating point, and \g{Delta} refers to the difference in $\left( \cdot \right)$ relative to the linearization point.

The resulting discrete-time, linear system is then augmented to include both error states as integrators for offset-free control, and delayed input states for the recycle valve.
One integrator for each output is added, as well as 40 delayed states per recycle valve (giving a total delay of \u{2}{s} for a sampling rate of \u{50}{ms}).
The augmented system has the following form:

\begin{equation}
  \begin{split}
    \gpio{xaug} & = 
    \ubrace{\begin{bmatrix}
      \ga{sys-mats} & \begin{bmatrix} \g{Bdelay} & 0\end{bmatrix} & 0 \\[0.5em]
      0 & \begin{bmatrix} 0 & \g{Adelay} \end{bmatrix} & 0\\[0.5em]
      0 & 0 & I_{n\ut{e}\times n\ut{e}}
    \end{bmatrix}}{\ga{augsys-mats}}
    \left( \ubrace{\begin{bmatrix}
      \gi{Delta} \gpi{xhat}\\
      \begin{bmatrix}
        {\gpi{udel}}{\left(1\right)}\\
        {\gpi{udel}}{\left(2\ldots\right)}\\
      \end{bmatrix}\\
      \gpi{integrator}
    \end{bmatrix}}{\gpi{xaug}} - 
    \begin{bmatrix}
      0\\
      \begin{bmatrix}
        u_{\text{r},k-1}\\
        0\\
      \end{bmatrix}\\
    \end{bmatrix}
    \right)\\
    & \qquad + 
    \ubrace{\begin{bmatrix}
      \g{Bnodelay}\\
      \g{Baug}\\
      0
    \end{bmatrix}}{\gb{augsys-mats}}
    \ubrace{\begin{bmatrix}
      u_{\text{r},k+i}\\
      \Delta T_{\text{d},k+i}
    \end{bmatrix}}{\gpi{deltau}}\\[0.5em]
    %
    % & = \gi{augsys-mats} \left( \gpi{xaug} - \vc{u}_{\text{r},k-1} \right) + \gii{augsys-mats} \gpi{deltau}\\[0.5em]
    % & = A_k^a \left(\Delta \vc{\hat{x}}_{k+i}^a - \vc{u}_{\text{r},k-1}\right) + B_k^a \Delta \vc{u}_{k+i}\\[0.5em]
    \gi{Delta} \g{ycurr} & = \ubrace{\begin{bmatrix}
      \gc{sys-mats} & 0 & I_{n\ut{e} \times n\ut{e}}
    \end{bmatrix}}{\gc{augsys-mats}}
    \g{xaug}\\[0.5em]
    % & = \begin{bmatrix}
      % C_k & 0 & I_{n\ut{e} \times n\ut{e}}
    % \end{bmatrix}
    % \vc{\hat{x}}_{k+i}^a,\\[0.5em]
    % \Delta \vc{y}_{k+i} & = C_k^a \Delta \vc{\hat{x}}_{k+i}^a\\[0.5em]
  \end{split}
  \label{eq:mpc:augmented-state-eqs}
\end{equation}

\noindent where \gpi{xaug} is the augmented state of the system, 
\gpi{xhat} contains the original states of the system, 
\gpi{udel} contains the delayed recycle valve inputs,
$u_{\text{r},k-1}$ gives the value of the recycle valve input about which the system was linearized,
\gpi{integrator} contains the integrator states,
$n\ut{e}$ gives the number of integrator states,
and \giv{augsys-mats} give the augmented, linearized model of the system.

The first component of \gpi{udel}  is multiplied by \g{Bdelay}, and must therefore be corrected by a factor $u_{\text{r},k-1}$, the value about which the system was linearized. 
The other components of \gpi{udel} are not corrected because they are only multiplied by \g{Adelay} which simply shifts them up a row.

The specific format of the augmented matrices depends on the system being modelled. For a single compressor, for example, the augmented matrices are as follows:

\begin{equation}
  \begin{split}
    \ga{augsys-mats} & =
    \begin{bmatrix}
      \ga{sys-mats} & \begin{bmatrix} B_{k,u\ut{r}} & 0 \end{bmatrix} & 0\\[0.5em]
      0 & \begin{bmatrix} 0 & I_{39\times 39}\\ 0 & 0\end{bmatrix} & 0\\[0.5em]
      0 & 0 & I_{2\times 2}
    \end{bmatrix}\\[1em]
%
    \gb{augsys-mats} & = 
    \begin{bmatrix}
      B_{k,T\ut{d}} & 0\\[0.5em]
      0 & \begin{bmatrix} 0_{39\times 1}\\ 1 \end{bmatrix} \\[0.5em]
      0 & 0
    \end{bmatrix}\\[1em]
%
    \gc{augsys-mats} & = \begin{bmatrix}
      \gc{sys-mats} & 0 & I_{n\ut{e} \times n\ut{e}}
    \end{bmatrix},
  \end{split}
\end{equation}

\noindent where $B_{k,u\ut{r}}$ and $B_{k,T\ut{d}}$ are the columns of \gb{sys-mats} corresponding to the recycle valve and torque inputs, respectively.

